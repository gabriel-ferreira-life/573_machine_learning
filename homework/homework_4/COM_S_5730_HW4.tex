\documentclass[11pt]{article}
\usepackage{color}
\usepackage{graphicx}
\usepackage{amsmath,amsthm,amssymb,multirow,paralist}
\usepackage[margin=0.8in]{geometry}
\usepackage[]{algorithm2e}
\usepackage{hyperref}
\usepackage{tikz,forest}
\usetikzlibrary{arrows.meta}

\providecommand{\abs}[1]{\left\vert#1\right\vert}
\providecommand{\norm}[1]{\left\Vert#1\right\Vert}

\begin{document}

\begin{center}
    {\Large \textbf{COM S 5730 Homework 4}}\\

    \linethickness{1mm}\line(1,0){498}

    \begin{enumerate}
\item Please put required code files and report into a
compressed file ``HW4\_FirstName\_LastName.zip''
\item Unlimited number of submissions are
allowed on Canvas and the latest one will be graded.
\item Due: \textbf{Tuesday Nov. 05, 2024 at 11:59pm.}
        \item {\color{red} No later submission is accepted.}
        \item Please read and follow submission instructions. No exception
              will be made to accommodate incorrectly submitted files/reports.
        \item All students are required to typeset their reports using
              latex. Overleaf
              (\url{https://www.overleaf.com/learn/latex/Tutorials}) can be a
              good start.
    \end{enumerate}

    \linethickness{1mm}\line(1,0){498}

\end{center}

%%%%%%%%%%%%%%%%%%%%%%%%%%%%%%%%%%%%%%%%%%%%%%%%%%%%%%%%%%%%%%%%%%%%%%%%%%%%%%%

%%%%%%%%%%%%%%%%%%%%%%%%%%%%%%%%%%%%%%%%%%%%%%%%%%%%%%%%%%%%%%%%%%%%%%%%%%%%%%%


\begin{enumerate}

    \item (40 points) \textbf{Hierarchical clustering}

    Use the similarity matrix in Table~\ref{tb:exp1} to perform
    (1) single (MIN) and (2) complete (MAX) link hierarchical
    clustering. Show each step with dendrogram and the
    corresponding similarity matrix update. The dendrogram should
    clearly show the order in which the points are merged.
    Suppose we choose to use 3 clusters, Show the cut in each
    final dendrogram.

    \begin{table}[ht]\label{tb:exp1}
        \centering
        \caption{Similarity matrix.}

        \begin{tabular}{ l| c | c | c | c | c}\hline
                   & \textbf{p1} & \textbf{p2} & \textbf{p3} & \textbf{p4} & \textbf{p5} \\ \hline
            \bf p1 & 1.00        & 0.10        & 0.41        & 0.55        & 0.35        \\
            \bf p2 & 0.10        & 1.00        & 0.64        & 0.47        & 0.98        \\
            \bf p3 & 0.41        & 0.64        & 1.00        & 0.44        & 0.85        \\
            \bf p4 & 0.55        & 0.47        & 0.44        & 1.00        & 0.76        \\
            \bf p5 & 0.35        & 0.98        & 0.85        & 0.76        & 1.00        \\
            \hline
        \end{tabular}
        % \vspace{-5pt}
    \end{table}

    \item (30 points) \textbf{K-Medians Clustering}
    
    The K-means algorithm can be summarized as below:
    \begin{enumerate}
        \item Select K points as the initial centroids.
        \item \textbf{repeat}
        \item \;\;\;\; Form K clusters by assigning all points to the closest centroid.
        \item \;\;\;\; Recompute the centroid of each cluster.
        \item \textbf{until} The centroids don't change.
    \end{enumerate}

    K-medians clustering is a variation of k-means clustering
    where it calculates the median for each cluster to determine
    its center instead of using the mean. Also, K-medians makes
    use of the Manhattan distance for points assignment.
    
    \begin{enumerate}
        \item (8 points) Please show the algorithm of K-medians
        in the above format.


        \item (6 points) Please explain how you will compute the median
        for each cluster.


        \item (6 points) Does K-medians help to avoid the outlier
        problem? Justify your answer.

    \end{enumerate}
    

    \item (30 points) \textbf{ Principal Components Analysis}
    
    Given three data points: $(-1, -1), (0,0), (1,1)$.

    \begin{enumerate}
        \item (10 points) Show the first Principal Component
        (actual vector) without using Eigendecomposition. Justify
        your answer.

        
        \item (10 points) If use the $1^{st}$ principle component
        to transform the data into 1-d space. What are the new
        data?


    \end{enumerate}



\end{enumerate}

\end{document}
