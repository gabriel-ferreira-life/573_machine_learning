\documentclass[11pt]{article}
\usepackage{color}
\usepackage{graphicx}
\usepackage{amsmath,amsthm,amssymb,multirow,paralist}
\usepackage[margin=0.8in]{geometry}
\usepackage[]{algorithm2e}
\usepackage{hyperref}
\usepackage{tikz,forest}
\usetikzlibrary{arrows.meta}
\usepackage{xcolor}
\usepackage{float}


\providecommand{\abs}[1]{\left\vert#1\right\vert}
\providecommand{\norm}[1]{\left\Vert#1\right\Vert}

\begin{document}

\begin{center}
    {\Large \textbf{COM S 5730 Homework 4}}\\

    \linethickness{1mm}\line(1,0){498}

    \begin{enumerate}
\item Please put required code files and report into a
compressed file ``HW4\_FirstName\_LastName.zip''
\item Unlimited number of submissions are
allowed on Canvas and the latest one will be graded.
\item Due: \textbf{Tuesday Nov. 05, 2024 at 11:59pm.}
        \item {\color{red} No later submission is accepted.}
        \item Please read and follow submission instructions. No exception
              will be made to accommodate incorrectly submitted files/reports.
        \item All students are required to typeset their reports using
              latex. Overleaf
              (\url{https://www.overleaf.com/learn/latex/Tutorials}) can be a
              good start.
    \end{enumerate}

    \linethickness{1mm}\line(1,0){498}

\end{center}

%%%%%%%%%%%%%%%%%%%%%%%%%%%%%%%%%%%%%%%%%%%%%%%%%%%%%%%%%%%%%%%%%%%%%%%%%%%%%%%

%%%%%%%%%%%%%%%%%%%%%%%%%%%%%%%%%%%%%%%%%%%%%%%%%%%%%%%%%%%%%%%%%%%%%%%%%%%%%%%


\begin{enumerate}

    \item (40 points) \textbf{Hierarchical clustering}

    Use the similarity matrix in Table~\ref{tb:exp1} to perform
    (1) single (MIN) and (2) complete (MAX) link hierarchical
    clustering. Show each step with dendrogram and the
    corresponding similarity matrix update. The dendrogram should
    clearly show the order in which the points are merged.
    Suppose we choose to use 3 clusters, Show the cut in each
    final dendrogram.

    \begin{table}[ht]\label{tb:exp1}
        \centering
        \caption{Similarity matrix.}

        \begin{tabular}{ l| c | c | c | c | c}\hline
                   & \textbf{p1} & \textbf{p2} & \textbf{p3} & \textbf{p4} & \textbf{p5} \\ \hline
            \bf p1 & 1.00        & 0.10        & 0.41        & 0.55        & 0.35        \\
            \bf p2 & 0.10        & 1.00        & 0.64        & 0.47        & 0.98        \\
            \bf p3 & 0.41        & 0.64        & 1.00        & 0.44        & 0.85        \\
            \bf p4 & 0.55        & 0.47        & 0.44        & 1.00        & 0.76        \\
            \bf p5 & 0.35        & 0.98        & 0.85        & 0.76        & 1.00        \\
            \hline
        \end{tabular}
        % \vspace{-5pt}
    \end{table}

    \begin{enumerate}
        \item \textbf{Single (Min)}
        \begin{itemize}
            \item \textbf{State 1:} Merge p2 and p5 (Highest similarity: 0.98).
\begin{table}[H]
        \centering
        \begin{minipage}{0.7\textwidth}  % Adjust width to control centering
            \centering
            \begin{minipage}{0.4\textwidth}
                \centering
                \label{tb:exp1}
                \begin{tabular}{ l| c | c | c | c | c}\hline
                           & \textbf{p1} & \textbf{p2} & \textbf{p3} & \textbf{p4} & \textbf{p5} \\ \hline
                    \bf p1 & 1.00        & 0.10        & 0.41        & 0.55        & 0.35        \\
                    \bf p2 & 0.10        & 1.00        & 0.64        & 0.47        &  \textcolor{red}{0.98}        \\
                    \bf p3 & 0.41        & 0.64        & 1.00        & 0.44        & 0.85        \\
                    \bf p4 & 0.55        & 0.47        & 0.44        & 1.00        & 0.76        \\
                    \bf p5 & 0.35        &  \textcolor{red}{0.98}        & 0.85        & 0.76        & 1.00        \\
                    \hline
                \end{tabular}
                \captionof{Similarity matrix: }{Before merging p2 and p5}
            \end{minipage}%
            \hfill
            \begin{minipage}{0.3\textwidth}
                \centering
                \includegraphics[width=\linewidth]{homework4/images/min_s1.png}
                \captionof{figure}{ $S_1$: Dendrogram after merging p2 and p5}
                \label{fig:your-image}
            \end{minipage}
        \end{minipage}
    \end{table}
    
    \item \textbf{State 2:} Merge \{p2, p5\} and p3 (Current highest similarity: 0.85)

        \begin{itemize}
        \item Updated Similarities:
        \begin{itemize}
            \item s(\{p2, p5\}, p1) = max(0.10, 0.35) = 0.35
            \item s(\{p2, p5\}, p3) = max(0.64, 0.85) = \textbf{0.85}
            \item s(\{p2, p5\}, p4) = max(0.47, 0.76) = 0.76 \\
        \end{itemize}
    \end{itemize}
    
\begin{table}[H]
        \centering
        \begin{minipage}{0.7\textwidth}  % Adjust width to control centering
            \centering
            \begin{minipage}{0.4\textwidth}
                \centering
                \label{tb:exp1}
                \begin{tabular}{ l| c | c | c | c}\hline
                           & \textbf{p1} & \textbf{\{p2\}}\textbf{\{p5\}}  & \textbf{p3} & \textbf{p4} \\ \hline
                    \bf p1 & 1.00        & 0.35         & 0.41        & 0.55               \\
                    \bf \{p2\}\{p5\} & 0.35        & 1.00         & \textcolor{red}{0.85}        & 0.76               \\
                    \bf p3 & 0.41        & \textcolor{red}{0.85}         & 1.00        & 0.44               \\
                    \bf p4 & 0.55        & 0.76         & 0.44        & 1.00               \\
                    \hline
                \end{tabular}
                \captionof{Similarity matrix: }{Before merging \{p2, p5\} and p3}
            \end{minipage}%
            \hfill
            \begin{minipage}{0.3\textwidth}
                \centering
                \includegraphics[width=\linewidth]{homework4/images/min_s2.png}
                \captionof{figure}{ $S_2$: Dendrogram after merging \{p2, p5\} and p3}
                \label{fig:your-image}
            \end{minipage}
        \end{minipage}
    \end{table}

        \item \textbf{State 3:} Merge \{p2, p3, p5\} and p4 (Next highest similarity: 0.76)

        \begin{itemize}
        \item Updated Similarities:
        \begin{itemize}
            \item s(\{p2, p3, p5\}, p1) = max(0.10, 0.41, 0.35) = 0.41
            \item s(\{p2, p3, p5\}, p4) = max(0.47, 0.44, 0.76) = \textbf{0.76} \\
        \end{itemize}
    \end{itemize}
    
\begin{table}[H]
        \centering
        \begin{minipage}{0.75\textwidth}  % Adjust width to control centering
            \centering
            \begin{minipage}{0.4\textwidth}
                \centering
                \label{tb:exp1}
                \begin{tabular}{ l| c | c | c}\hline
                           & \textbf{p1} & \textbf{\{p2\}}\textbf{\{p5\}}\textbf{\{p3\}} & \textbf{p4} \\ \hline
                    \bf p1                 & 1.00        & 0.41        & 0.55               \\
                    \bf \{p2\}\{p5\}\{p3\} & 0.41        & 1.00        & \textcolor{red}{0.76}                           \\
                    \bf p4                 & 0.55        & \textcolor{red}{0.76}        & 1.00               \\
                    \hline
                \end{tabular}
                \captionof{Similarity matrix: }{Before merging \{p2, p3, p5\} and p4}
            \end{minipage}%
            \hfill
            \begin{minipage}{0.28\textwidth}
                \centering
                \includegraphics[width=\linewidth]{homework4/images/min_s3.png}
                \captionof{figure}{ $S_3$: Dendrogram after merging \{p2, p3, p5\} and p4}
                \label{fig:your-image}
            \end{minipage}
        \end{minipage}
    \end{table}

            \item \textbf{State 4:} Merge {p2, p3, p4, p5} and p1 (Similarity: 0.55)

        \begin{itemize}
        \item Updated Similarities:
        \begin{itemize}
            \item s(\{p2, p3, p4, p5\}, p1) = max(0.10, 0.41, 0.55, 0.35) = \textbf{0.55}\\
        \end{itemize}
    \end{itemize}
    
\begin{table}[H]
        \centering
        \begin{minipage}{0.85\textwidth}  % Adjust width to control centering
            \centering
            \begin{minipage}{0.5\textwidth}
                \centering
                \label{tb:exp1}
                \begin{tabular}{ l| c | c}\hline
                           & \textbf{p1} & \textbf{\{p2\}}\textbf{\{p5\}}\textbf{\{p3\}}\textbf{\{p4\}} \\ \hline
                    \bf \hspace{1.5cm} p1                 & 1.00        & \textcolor{red}{0.55} \\
                    \bf \{p2\}\{p5\}\{p3\}\{p4\} & \textcolor{red}{0.55}        & 1.00 \\
                    \hline
                \end{tabular}
                \captionof{Similarity matrix: }{Before merging \{p2, p3, p4, p5\} and p1}
            \end{minipage}%
            \hfill
            \begin{minipage}{0.28\textwidth}
                \centering
                \includegraphics[width=\linewidth]{homework4/images/min_s4.png}
                \captionof{figure}{ $S_4$: Dendrogram after merging \{p2, p3, p4, p5\} and p1}
                \label{fig:your-image}
            \end{minipage}
        \end{minipage}
    \end{table}
    
    \end{itemize}

    \textbf{Final Clusters:}
    \begin{itemize}
        \item \textbf{Cluster 1:} {p1}
        \item \textbf{Cluster 2:} {p4}
        \item \textbf{Cluster 3:} {p2, p3, p5}\\
    \end{itemize}

    \item \textbf{Complete (Max)}
        \begin{itemize}
            \item \textbf{State 1:} Merge p2 and p5 (Highest similarity: 0.98).
\begin{table}[H]
        \centering
        \begin{minipage}{0.7\textwidth}  % Adjust width to control centering
            \centering
            \begin{minipage}{0.4\textwidth}
                \centering
                \label{tb:exp1}
                \begin{tabular}{ l| c | c | c | c | c}\hline
                           & \textbf{p1} & \textbf{p2} & \textbf{p3} & \textbf{p4} & \textbf{p5} \\ \hline
                    \bf p1 & 1.00        & 0.10        & 0.41        & 0.55        & 0.35        \\
                    \bf p2 & 0.10        & 1.00        & 0.64        & 0.47        &  \textcolor{red}{0.98}        \\
                    \bf p3 & 0.41        & 0.64        & 1.00        & 0.44        & 0.85        \\
                    \bf p4 & 0.55        & 0.47        & 0.44        & 1.00        & 0.76        \\
                    \bf p5 & 0.35        &  \textcolor{red}{0.98}        & 0.85        & 0.76        & 1.00        \\
                    \hline
                \end{tabular}
                \captionof{Similarity matrix: }{Before merging \{p2\} and \{p5\}}
            \end{minipage}%
            \hfill
            \begin{minipage}{0.3\textwidth}
                \centering
                \includegraphics[width=\linewidth]{homework4/images/max_s1.png}
                \captionof{figure}{ $S_1$: Dendrogram after merging \{p2\} and \{p5\}}
                \label{fig:your-image}
            \end{minipage}
        \end{minipage}
    \end{table}
    
    \item \textbf{State 2:} Merge \{p2, p5\} and p3 (Current highest similarity: 0.64)

        \begin{itemize}
        \item Updated Similarities:
        \begin{itemize}
            \item s(\{p2, p5\}, p1) = min(0.10, 0.35) = 0.10
            \item s(\{p2, p5\}, p3) = min(0.64, 0.85) = \textbf{0.64}
            \item s(\{p2, p5\}, p4) = min(0.47, 0.76) = 0.47 \\
        \end{itemize}
    \end{itemize}
    
\begin{table}[H]
        \centering
        \begin{minipage}{0.7\textwidth}  % Adjust width to control centering
            \centering
            \begin{minipage}{0.4\textwidth}
                \centering
                \label{tb:exp1}
                \begin{tabular}{ l| c | c | c | c}\hline
                           & \textbf{p1} & \textbf{\{p2\}}\textbf{\{p5\}}  & \textbf{p3} & \textbf{p4} \\ \hline
                    \bf p1 & 1.00        & 0.10         & 0.41        & 0.55               \\
                    \bf \{p2\}\{p5\} & 0.10        & 1.00         & \textcolor{red}{0.64}        & 0.47               \\
                    \bf p3 & 0.41        & \textcolor{red}{0.64}         & 1.00        & 0.44               \\
                    \bf p4 & 0.55        & 0.47         & 0.44        & 1.00               \\
                    \hline
                \end{tabular}
                \captionof{Similarity matrix: }{Before merging \{p2, p5\} and \{p3\}}
            \end{minipage}%
            \hfill
            \begin{minipage}{0.3\textwidth}
                \centering
                \includegraphics[width=\linewidth]{homework4/images/max_s2.png}
                \captionof{figure}{ $S_2$: Dendrogram after merging \{p2, p5\} and \{p3\}}
                \label{fig:your-image}
            \end{minipage}
        \end{minipage}
    \end{table}

        \item \textbf{State 3:} Merge p1 and p4 (Next highest similarity: 0.55)

        \begin{itemize}
        \item Updated Similarities:
        \begin{itemize}
            \item s(\{p2, p3, p5\}, p1) = min(0.10, 0.41, 0.35) = 0.10
            \item s(\{p2, p3, p5\}, p4) = min(0.47, 0.44, 0.76) = 0.44
        \end{itemize}
    \end{itemize}
    
\begin{table}[H]
        \centering
        \begin{minipage}{0.75\textwidth}  % Adjust width to control centering
            \centering
            \begin{minipage}{0.4\textwidth}
                \centering
                \label{tb:exp1}
                \begin{tabular}{ l| c | c | c}\hline
                           & \textbf{p1} & \textbf{\{p2\}}\textbf{\{p5\}}\textbf{\{p3\}} & \textbf{p4} \\ \hline
                    \bf p1                 & 1.00       & 0.10        & \textcolor{red}{0.55}  \\
                    \bf \{p2\}\{p5\}\{p3\} & 0.10        & 1.00        & 0.44                           \\
                    \bf p4                 & \textcolor{red}{0.55}        & 0.44        & 1.00               \\
                    \hline
                \end{tabular}
                \captionof{Similarity matrix: }{Before merging \{p1\} and \{p4\}}
            \end{minipage}%
            \hfill
            \begin{minipage}{0.28\textwidth}
                \centering
                \includegraphics[width=\linewidth]{homework4/images/max_s3.png}
                \captionof{figure}{ $S_1$: Dendrogram after merging \{p1\} and \{p4\}}
                \label{fig:your-image}
            \end{minipage}
        \end{minipage}
    \end{table}

            \item \textbf{State 4:} Merge {p1, p4} and {p2, p3, p5} (Similarity: 0.10)

        \begin{itemize}
        \item Updated Similarities:
        \begin{itemize}
            \item s(\{p1, p4\}, \{p2, p3, p5\}) = min(0.10, 0.41, 0.35, 0.47, 0.44, 0.76) = 0.10
        \end{itemize}
    \end{itemize}
    
\begin{table}[H]
        \centering
        \begin{minipage}{0.85\textwidth}  % Adjust width to control centering
            \centering
            \begin{minipage}{0.5\textwidth}
                \centering
                \label{tb:exp1}
                \begin{tabular}{ l| c | c}\hline
                           & \textbf{\{p1\}}\textbf{\{p4\}} & \textbf{\{p2\}}\textbf{\{p5\}}\textbf{\{p3\}} \\ \hline
                    \bf \{p1\}\{p4\}                & 1.00        & \textcolor{red}{0.10} \\
                    \bf \{p2\}\{p5\}\{p3\} & \textcolor{red}{0.10}        & 1.00 \\
                    \hline
                \end{tabular}
                \captionof{Similarity matrix: }{Before merging \{p1, p4\}, \{p2, p3, p5\}}
            \end{minipage}%
            \hfill
            \begin{minipage}{0.28\textwidth}
                \centering
                \includegraphics[width=\linewidth]{homework4/images/max_s4.png}
                \captionof{figure}{ $S_1$: Dendrogram after merging \{p1, p4\}, \{p2, p3, p5\}}
                \label{fig:your-image}
            \end{minipage}
        \end{minipage}
    \end{table}
    
    \end{itemize}

    \textbf{Final Clusters:}
    \begin{itemize}
        \item \textbf{Cluster 1:} {p2, p5}
        \item \textbf{Cluster 2:} {p3}
        \item \textbf{Cluster 3:} {p1, p4}\\
    \end{itemize}
    \end{enumerate}

    \item (30 points) \textbf{K-Medians Clustering}
    
    The K-means algorithm can be summarized as below:
    \begin{enumerate}
        \item Select K points as the initial centroids.
        \item \textbf{repeat}
        \item \;\;\;\; Form K clusters by assigning all points to the closest centroid.
        \item \;\;\;\; Recompute the centroid of each cluster.
        \item \textbf{until} The centroids don't change.
    \end{enumerate}

    K-medians clustering is a variation of k-means clustering
    where it calculates the median for each cluster to determine
    its center instead of using the mean. Also, K-medians makes
    use of the Manhattan distance for points assignment.
    
    \begin{enumerate}
        \item (8 points) Please show the algorithm of K-medians
        in the above format.\\

            The K-medians algorithm can be summarized as below:
            \begin{enumerate}
                \item Select K points as the initial centroids.
                \item \textbf{repeat}
                \item \;\;\;\; Form K clusters by assigning all points to the closest centroid using Manhattan distance as the distance metric.
                \item \;\;\;\; Recompute the centroid of each cluster by calculating the median of the data points in that cluster.
                \item \textbf{until} The centroids don't change.\\
            \end{enumerate}

        

        \item (6 points) Please explain how you will compute the median
        for each cluster.\\

        Given a cluster $K$ with $X$ data points $X = [x_1, x_2, ... , x_n]$ where $x_i \in \mathbb{R}^d$, the centroid recomputation for $K$ will be based on the median of each dimension.\\

        For each dimension $d$, sort the values of $X$ along $d$. The median, $Med(X^d)$, is calculated as:
        
        \[Med(X^d) = \begin{cases}
         X [\frac{n+1}{2}] \text{ if n is odd} \\\\
         \frac{X[\frac{n}{2}] + X[\frac{n}{2} + 1]}{2}] \text{ if n is even}
        \end{cases}\]

        Compute the median independently for each dimension $d$, resulting in a new centroid \\
        $(Med(X^1), Med(X^2),..., Med(X^d))$ for the cluster.\\


        \item (6 points) Does K-medians help to avoid the outlier
        problem? Justify your answer.\\

        Yes, K-medians helps to avoid the influence of outliers because the median is not as affected by outliers as the mean. In K-medians clustering, using the median to determine the cluster centroid makes it more robust to extreme values, whereas K-means, which relies on the mean, can have its centroids skewed by outliers.

    \end{enumerate}
    

    \item (30 points) \textbf{ Principal Components Analysis}
    
    Given three data points: $(-1, -1), (0,0), (1,1)$.

    \begin{enumerate}
        \item (10 points) Show the first Principal Component
        (actual vector) without using Eigendecomposition. Justify
        your answer.

        \begin{itemize}
            \item \textbf{Step 1: Calculate the mean}

            \[(\frac{-1+0+1}{3}, \frac{-1+0+1}{3}) = (0,0)\]

            \item \textbf{Step 2: Subtract the mean out of each point}

            \[\textbf{(-1, -1): } (-1, -1) - (0, 0) = (-1, -1)\]
            \[\textbf{(0,0): } (0,0) - (0, 0) = (0,0)\]
            \[\textbf{(1, 1): } (1, 1) - (0, 0) = (1, 1)\]

            \item \textbf{Step 3: Find vector along the maximum distance in the positive direction}

            \[\text{(1, 1)  vector along maximum variance.}\]

            \item \textbf{Step 4: Normalize vector by dividing it by its magnitude to have a unit vector for direction.}

            \[\textbf{(1, 1) Magnitude: } \sqrt{1^2 + 1^2} = \sqrt{2}\]
    
            \[\textbf{Normalized Vector: } (1, 1) \div \sqrt{2} = [\frac{1}{\sqrt{2}}, \frac{1}{\sqrt{2}}]\]\\

            \item The $1^{st}$ principle component is $[\frac{1}{\sqrt{2}}, \frac{1}{\sqrt{2}}]^T$.

        \end{itemize}
        
        \item (10 points) If use the $1^{st}$ principle component
        to transform the data into 1-d space. What are the new
        data?

        \begin{itemize}
            \item \textbf{Step 5: Use the $1^{st}$ principle component
        to transform the data into 1-d space}

            \[\textbf{(-1, -1): } (-1, -1) \cdot [\frac{1}{\sqrt{2}}, \frac{1}{\sqrt{2}}]^T = \frac{-2}{\sqrt{2}} = -\sqrt{2}\]
            \[\textbf{(0,0): } (0,0) \cdot [\frac{1}{\sqrt{2}}, \frac{1}{\sqrt{2}}]^T = 0\]
            \[\textbf{(1, 1): } (1, 1) \cdot [\frac{1}{\sqrt{2}}, \frac{1}{\sqrt{2}}]^T = \frac{2}{\sqrt{2}} = \sqrt{2}\]

        \textbf{The new data is:} $-\sqrt{2}, 0, \sqrt{2}$
            
        \end{itemize}
    

    \end{enumerate}



\end{enumerate}

\end{document}
