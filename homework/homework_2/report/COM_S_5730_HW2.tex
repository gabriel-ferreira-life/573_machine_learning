\documentclass[11pt]{article}
\usepackage{color}
\usepackage{amsmath,amsthm,amssymb,multirow,paralist}
\usepackage[margin=0.8in]{geometry}
\usepackage{hyperref}

\begin{document}

\begin{center}
{\Large \textbf{COM S 5730 Homework 2}}\\

\linethickness{1mm}\line(1,0){498}

\begin{enumerate}
\item Please put required code files and report into a
compressed file ``HW2\_FirstName\_LastName.zip''
\item Unlimited number of submissions are
allowed on Canvas and the latest one will be graded.
\item Due: \textbf{Tuesday Oct. 01, 2024 at 11:59pm.}
\item {\color{red} No later submission is accepted.}
\item Please read and follow submission instructions. No exception
will be made to accommodate incorrectly submitted files/reports.
\item All students are required to typeset their reports using
latex. Overleaf
(\url{https://www.overleaf.com/learn/latex/Tutorials}) can be a
good start.
\end{enumerate}

\linethickness{1mm}\line(1,0){498}

\end{center}

%%%%%%%%%%%%%%%%%%%%%%%%%%%%%%%%%%%%%%%%%%%%%%%%%%%%%%%%%%%%%%%%%%%%%%%%%%%%%%%

%%%%%%%%%%%%%%%%%%%%%%%%%%%%%%%%%%%%%%%%%%%%%%%%%%%%%%%%%%%%%%%%%%%%%%%%%%%%%%%


\begin{enumerate}

\item (20 points) Consider the toy data set $\{([0, 0], -1),
([2, 2], -1), ([2, 0], +1)\}$. Set up the dual problem for
the toy data set. Then, solve the dual problem and compute
$\alpha^*$, the optimal Lagrange multipliers. (Note that there will
be three weights $\boldsymbol w = [w_0, w_1, w_2]$ by considering
the bias.)

\item (20 points) In a separable case, when a multiplier
$\alpha_i > 0$, its corresponding data point $(\boldsymbol x_i,
y_i)$ is on the boundary of the optimal separating hyperplane
with $y_i(\boldsymbol w^T \boldsymbol x_i) = 1$.

Show that the inverse is not True. Namely, it is possible that
$\alpha_i = 0$ and $(\boldsymbol x_i, y_i)$ is on the boundary
satisfying $y_i(\boldsymbol w^T \boldsymbol x_i) = 1$.

[Hint: Consider a toy data set with two positive examples at
([0,0], +1) and ([1, 0], +1), and one negative example at ([0,
1], -1).] (Note that there will be three weights $\boldsymbol w =
[w_0, w_1, w_2]$ by considering the bias.)

\item (20 points) \textbf{Non-separable Case SVM:} In our lecture, we compared the hard-margin SVM and soft-margin SVM.
Prove that the dual problem of soft-margin SVM is almost
identical to the hard-margin SVM, except that $\alpha_i$s are
now bounded by $C$ (tradeoff parameter).

\item (20 points) \textbf{Kernel Function:} A function $K$
computes $K(\boldsymbol x_i, \boldsymbol x_j) =
-\boldsymbol{x_i}^T \boldsymbol x_j $. Is this function a valid
kernel function for SVM? Prove or disprove it.



\item (20 points) \textbf{Support Vector Machine for Handwritten
Digits Recognition}: You need to use the software package
scikit-learn
\href{https://scikit-learn.org/stable/modules/svm.html}{https://scikit-learn.org/stable/modules/svm.html}
to finish this assignment. We will use ``svm.SVC()'' to create a
svm model. The handwritten digits files are in the ``data''
folder: train.txt and test.txt. The starting code is in the
``code'' folder. In the data file, each row is a data example.
The first entry is the digit label (``1'' or ``5''), and the next
256 are grayscale values between -1 and 1. The 256 pixels
correspond to a $16\times16$ image. You are expected to implement
your solution based on the given codes. The only file you need to
modify is the ``solution.py'' file. You can test your solution by
running ``main.py'' file. Note that code is provided to compute a
two-dimensional feature (symmetry and average intensity) from
each digit image; that is, each digit image is represented by a
two-dimensional vector. These features along with the
corresponding labels should serve as inputs to your solution
functions.

\begin{enumerate}
\item (5 points) Complete the \textbf{svm\_with\_diff\_c()}
function. In this function, you are asked to try different values
of cost parameter c.
\item (10 points) Complete the \textbf{svm\_with\_diff\_kernel()}
function. In this function, you are asked to try different
kernels (linear, polynomial and radial basis function kernels).
\item (5 points) Summarize your observations from (a) and (b)
into a short report. In your report, please report the accuracy
result and total support vector number of each model. A briefly
analysis based on the results is also needed. For example, how
the number of support vectors changes as parameter value changes
and why.
\end{enumerate}


\end{enumerate}

\end{document}
